\capitulo{4}{Técnicas y herramientas}

Esta parte de la memoria tiene como objetivo presentar las técnicas metodológicas y las herramientas de desarrollo que se han utilizado para llevar a cabo el proyecto. Se comentará de manera breve, las diferentes opciones y la razón porque estás fueron descartadas.
 
\section{Control de versiones}

\subsection{Opciones elegidas}

\subsubsection{Git}
~\href{https://git-scm.com//}{Git} es un software de control de versiones, pensado para trabajar con gran cantidad de archivos, con el fin de llevar el registro de los cambios y coordinar a las personas que los comparten. Es gratuito y de código abierto.

Me he decantado por él, porque lo usé durante las prácticas curriculares y extracurriculares de forma intensa, mediante la herramienta Git Bash, ya que mediante comando me siento más cómodo que con una interfaz. Aunque también la dispone mediante el comando \emph{gitk}.

\subsubsection{GitHub}\label{github}
~\href{https://github.com/}{Github} es una plataforma web, recientemente comprada por Microsoft, usada para el control de versiones con las funciones de Git. Entre las diferentes herramientas a destacar: wiki para cada uno de los proyectos, gráficos, funcionalidades de red social, gestor de proyectos, entre otras.

Se escogió esta porque la hemos usado durante el grado y es de sobra conocida. Además ofrece la posibilidad de integración con la herramienta Zenhub~\pageref{zenhub}, para la gestión del proyecto, teniendo las dos cosas centralizadas en el mismo lugar, lo que facilita el proceso de desarrollo.

\subsection{Alternativas descartadas}

\subsubsection{Gitlab}
~\href{https://gitlab.com/}{Gitlab}es igual que Github pero de código abierto ya que tiene licencia MIT. También lo usé en la empresa durante las prácticas pero fue descartado porque me parece de menor calidad. En cuanto a espacio este tiene 10 GB a favor, en contra del 1 GB de Github.

\subsubsection{Bitbucket}
~\href{https://bitbucket.org/product//}{Bitbucket} es otra web más para el control de versiones. Esta enfocado más a la empresa privada, ya que se suele integrar muy bien con otras herramientas de gestión de proyectos. Se descarto por el poco uso que he tenido con este.

\subsubsection{Extensiones Visual Studio Code}
Hay una gran cantidad de herramientas para el control de versiones en la tienda de este editor de código, puede ser práctico, pero las interfaces pueden ser liosas. Por eso me gusta más mediante comando, descartando rápidamente esta opción.


\section{Gestión de proyecto}

\subsection{Opción elegida}

\subsubsection{Zenhub}\label{zenhub}
~\href{https://bitbucket.org/product//}{Zenhub} es una herramienta de gestión de proyectos, que viene por defecto integrada en Github~\pageref{github}, lo que implica a usarla sin pensarlo mucho. En mi caso lo que más usé fue el tablero de \emph{kanban} donde poder ver las \emph{issues} que me planificaba para cada \emph{sprint}. Ofrece diferentes tipos de gráficos, el que mejor se encajaba a la planificación fue de \emph{burndown}, ya que me permite ver lo ideal del proyecto y la progresión que llevo. No me gustó que las tareas las cierra por días en vez de por horas.

\subsection{Alternativas descartadas}

\subsubsection{Jira}
~\href{https://www.atlassian.com/es/software/jira}{Jira} es una herramienta web propietaria, para el control, seguimiento de errores e incidencias, dentro de la gestión de proyectos. La conozco porque la usé durante las prácticas curriculares, me gustaba mucho, pero no me parece cómoda para lo que quería hacer. Ya que esta herramienta esta orientada en la mejora de procesos en la empresa, que al desarrollo de software.

\subsubsection{Trello}
~\href{https://trello.com/es}{Trello} es una herramienta de gestión de proyectos, con interfaz móvil y web. Usa el sistema kanban como Zenhub~\pageref{zenhub} y tiene integración con Github~\pageref{github}, pero me parece que no encaja en proyectos unipersonales, como era mi caso. Ya que el enfoque es más colaborativo, con grandes grupos de trabajo, donde es necesario compartir documentos con los requisitos, etc ...

\section{Entorno de desarrollo integrado (IDE)}

\subsection{Opción elegida}

\subsubsection{Visual Studio Code}
~\href{https://code.visualstudio.com/}{Visual Studio Code} es un editor de código fuente desarrollado por Microsoft, cuya licencia es MIT. Esta es totalmente gratuita, con una cantidad enorme de extensiones o \emph{pluggins}, que minimizan esfuerzos, además de la integración con git, que permite ver los cambios en tiempo real. 

Esta facilidad de uso hicieron que me decantase por él.

\subsection{Alternativas descartadas}

\subsubsection{Android Studio}
~\href{https://developer.android.com/studio}{Android Studio} es la herramienta oficial de desarrollo de aplicaciones móviles para el sistema operativo Android. Este sustituye a Eclipse como el IDE, cuya licencia es Apache 2.0.

Este IDE se descarta para la creación de código, pero no para las máquinas virtuales.
