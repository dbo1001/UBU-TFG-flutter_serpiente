\capitulo{4}{Técnicas y herramientas}

Esta parte de la memoria tiene como objetivo presentar las técnicas metodológicas y las herramientas de desarrollo que se han utilizado para llevar a cabo el proyecto. Se comentará de manera breve, las diferentes opciones y la razón porque estás fueron descartadas.
 
\section{Control de versiones}

\subsection{Opciones elegidas}

\subsubsection{GitHub}
~\href{https://git-scm.com//}{Git} es un software de control de versiones, pensado para trabajar con gran cantidad de archivos, con el fin de llevar el registro de los cambios y coordinar a las personas que los comparten. Es gratuito y de código abierto.

Me he decantado por él, porque lo usé durante las prácticas curriculares y extracurriculares de forma intensa, mediante la herramienta Git Bash~\pageref{gitbash}, ya que mediante comando me siento más cómodo que con una interfaz.

\subsubsection{GitHub}
~\href{https://github.com/}{Github} es una plataforma web, recientemente comprada por Microsoft, usada para el control de versiones con las funciones de Git. Entre las diferentes herramientas a destacar: wiki para cada uno de los proyectos, gráficos, funcionalidades de red social, gestor de proyectos, entre otras.

Se escogió esta porque la hemos usado durante el grado y es de sobra conocida. Además ofrece la posibilidad de integración con la herramienta Zenhub~\pageref{zenhub}, para la gestión del proyecto, teniendo las dos cosas centralizadas en el mismo lugar, lo que facilita el proceso de desarrollo.