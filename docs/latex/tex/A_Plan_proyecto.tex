\apendice{Plan de Proyecto Software}

\section{Introducción}
En este apartado trabajaremos sobre la planificación del proyecto, de tal manera que se pueda definir, identificar y programar las actividades específicas que se requieren para realizar las tareas del mismo. 

La evolución temporal es una de las partes más importantes de todo el proceso de desarollo, ya que una mala planificación, puede hacer que el proyecto sufra retrasos, de tal manera, que no se llegue a la fecha de entrega prevista, lo que supone un coste, en este caso, el suspenso, pero también el económico, por las horas y recursos destinados a tal fin.

 En esta fase es muy importante que para cada una de las tareas sepamos el tiempo que durará aproximadamente, quien es el encargado de hacer la tarea y el dinero que supone hacerla. Por lo que invertir tiempo en la estimación de las horas cada una de las tareas, ayuda a identificar las irregularidades en el futuro.

La viabilidad pone el foco en el coste económico del proyecto como de la parte legal. Es decir, es un reactivo limitante, sobre todo el coste.

\section{Planificación temporal}
El método de trabajo para hacer el seguimiento y la planificación del mismo es Scrum, que pretende realizar una gestión ágil del proyecto. Se basa en \emph{spints}, la duración de estos suele rondar entre los siete y quince días. Dependiendo de los requisitos del proyecto, número de integrantes y del tiempo disponible, se maneja esta horquilla temporal, en mi caso por la falta de tiempo, decidí hacerlos de 5 días. 

Estos \emph{sprints} se basan en una reunión donde se ponen todas las tareas que se tiene como objetivo realizar, en mi caso eran las reuniones con los tutores del trabajo de fin de grado. Además cada día se tiene que hacer el \emph{daily meetting}, pero como el proyecto es unipersonal, no es necesario. Por lo que se puede afirmar que se ha seguido la filosofía ágil.

Todo ello esta integrado en un proceso incremental de desarrollo del producto, ya que en cada uno de los \emph{spirnts} se producen las mejoras planificadas.

Aclarar que la estimación del tiempo se realiza mediante los \emph{story points}, que indican la complejidad de la tarea a realizar. Esta herramienta nos la aporta ZenHub, con la siguiente relación según el coste temporal:

\tablaSmall{Equivalencias \emph{Story Points} y tiempo estimado}{c c }{StoryPoints/tiempo}
{ \multicolumn{1}{l}{Story Points} & Estimación temporal \\}{ 
	1            & 1 hora              \\ 
	2            & 2 horas           \\ 
	3            & 3 horas             \\ 
	4            & 4 horas           \\ 
	5            & 5 horas             \\ 
	6            & 6 horas           \\ 
	7            & 7 horas             \\ 
	8            & 8 horas             \\ 
	9            & 9 horas             \\ 
}

Algo que define mucho la duración de los \emph{sprints} a 5 días, es porque el proyecto en el que estaba, no me funcionaba y al final decidí por hacer este otro. Esto implica tener mayor cantidad de \emph{sprints}, con un incremento del aplicación mayor, ya que el margen temporal hasta la entrega era de un mes. 

A continuación se detallan cada uno de los \emph{sprints} realizados durante el proyecto:

\subsection{Sprint 1 (22/06/20 - 26/096/20)}\label{sprint-1-220620---260620}

El inicio del proyecto fue a través de correo, explicando la situación a mis tutores, de que lo que estaba haciendo no me funcionaba, que estaba verde y que si cabía la posibilidad de hacer otro trabajo. Me dieron el visto bueno, pero que tenía que hacer algo diferenciador, ya que no vale cualquier cosa. Por lo que en la reunión de cierre de \emph{sprint} se comentaría como centrar la aplicación. 

Entonces debido a la escasez temporal, me decido a hacer una aplicación en \emph{Flutter}, ya que se puede hacer algo de calidad de manera ágil.

Los objetivos en este \emph{sprint} inicial fueron:

\begin{itemize}
\item Documentar como hacer aplicaciones en \emph{Flutter}.
\item Cursar un curso en \href{https://www.udemy.com/course/flutter-primeros-pasos/}{Udemy}
\item Crear el repositorio.
\item Implementar el cuerpo de la aplicación.
\item Investigar sobre \emph{apis} que ofrece el \emph{framework}.
\end{itemize}

Todas las \emph{issues} realizadas para este \emph{sprint}, están disponibles en \href{https://github.com/scc0034/flutter_serpiente/milestone/1?closed=1}{Sprint 1}

La estimación fue de 54 horas, pero que finalmente se destinaron 45,5 horas, debido en gran parte a la reutilización de código del curso, aunque el curso me duró más de lo estimado.

\imagen{burndowns/sp1.png}{Sprint 1.}


\section{Estudio de viabilidad}

Es esta parte se pretende analizar el coste / beneficio, como el apartado legal, en todo el proceso de desarrollo, en el caso de que se hubiera tenido que realizar en un entorno real. 


\subsection{Viabilidad económica}
La estructura de mi trabajo, es la de un proyecto con precio cerrado, es decir, no soy un trabajador asalariado, sino autónomo, que recibe un encargo, en este caso por parte de la Universidad. La definición que encaja para esto es la de \emph{freelancer}. Por lo que el coste dependerá de la estimación inicial de las horas del proyecto, incluyendo un porcentaje de beneficios.

Otra de las vías para obtener ingresos es mediante los anuncios que nos ofrece \emph{Google Ad}, a través de diferentes tipos de \emph{banners} y videos, esperamos tener un retorno de dinero. Ya que la idea es que la aplicación este disponible de manera gratuita, en la \emph{play store} para todos aquellos que se la quieran descargar. Aunque esto este disponible, para contabilizar los costes es algo despreciable, ya que es complicado tener una gran repercusión en las primeras etapas de proyecto, o incluso una vez finalizado. 	

\subsubsection{Coste hardware}
Se desglosa el coste de los dispositivos usados para la implementación, suponiendo que la amortización del pc de sobremesa es aproximadamente de 5 años y de 2 años para el \emph{smartphone}, con la duración del proyecto de 1 mes.

\begin{table}[H]
	\begin{center}
		\begin{tabular}{16}
			\hline
			Concepto                        & Coste (€) & Coste amortización (€) \\ \hline
			PC sobremesa				    & 1200      & 20						\\
			Dispositivo móvil			    & 450       & 18,75						\\ \hline
			Coste total mensual             & 1558,80  	& 1200						\\ \hline
		\end{tabular}
	\end{center}
\end{table}

\subsubsection{Coste software}
El coste de los librerías, \emph{cloud services}, entornos de desarrollo, licencias, cursos, máquinas virtuales, entre otros, han sido de uso gratuito, dando lugar a un coste software de 0 euros.



\subsection{Viabilidad legal}


