\apendice{Especificación de Requisitos}

\section{Introducción}
En este apéndice se detallan los requisitos del proyecto, como los funcionales y no funcionales. La finalidad es hacer de intermediario entre el cliente y los programadores, con el objetivo de ayudar a entender, comprender y analizar la aplicación. 

\section{Objetivos generales}
Este trabajo final de grado focaliza la aplicación mediante dos puntos de vista totalmente diferentes, dependiendo de las situaciones futuras a las que se someta la aplicación, siempre englobado en el marco de que es un producto que nos encarga la universidad de Burgos, como cliente final.

\begin{itemize}
	\item Colección de videojuegos. Tener una gran cantidad de estos mismos, para lograr sacar un rendimiento económico gracias a lo publicidad, por lo que es necesario que la aplicación tenga una gran repercusión en el mercado. Esto implica que a futuro se debe de hacer una inversión en publicidad y \emph{marketing}.
	\item \emph{Porfolio} \cite{wiki:portafolio} o escaparate con el que mostrar las capacidades técnicas, herramientas usadas o lenguajes de programación aprendidos, ante los equipos de recursos humanos en las empresas, llegado el momento que tenga que buscar trabajo. 
\end{itemize}

\section{Catalogo de requisitos}

\section{Especificación de requisitos}


