\apendice{Documentación técnica de programación}

\section{Introducción}
En este anexo se describe la documentación técnica de programación para este proyecto. Incluye los primeros pasos que son la instalación del proyecto, la estructura de la aplicación o finalmente como compilarlo, desplegarlo o los diferentes tipos de configuraciones realizados. La idea es poder facilitar a los futuros desarrolladores una guía con la que poder comenzar, en el caso de que quisieran continuar con el trabajo.

\section{Estructura de directorios}
El repositorio se encuentra alojado en \href{https://github.com/scc0034/flutter_serpiente}{Github}. La estructura de ficheros que sigue es la siguiente:

\begin{itemize}
\item \textbf{./}
Directorio raíz del que cuelgan todas los demás ficheros. Este contiene uno de los archivos más importantes, que es es \emph{pubspec.yalm}. Este archivos se usa para hacer las importaciones de las nuevas funcionalidades que queramos dar a nuestra aplicación.

\begin{itemize}
	\item \textbf{a}: 
	\item \textbf{b}: 
	\item \textbf{c}: 
\end{itemize}

Tambien contiene los siguientes archivos:
\begin{itemize}
	\item \textbf{a}: 
	\item \textbf{b}:
	\item \textbf{c}:
\end{itemize}


\end{itemize}

\section{Manual del programador}

\section{Compilación, instalación y ejecución del proyecto}

\section{Pruebas del sistema}
