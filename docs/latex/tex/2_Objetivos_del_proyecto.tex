\capitulo{2}{Objetivos del proyecto}

%% Comentario de la plantilla
%%Este apartado explica de forma precisa y concisa cuales son los objetivos que se persiguen con la realización del proyecto. Se puede distinguir entre los objetivos marcados por los requisitos del software a construir y los objetivos de carácter técnico que plantea a la hora de llevar a la práctica el proyecto.

A continuación, se detallarán los objetivos que han motivado la realización de este proyecto así como los resultados que se desean conseguir.

\section{Objetivos generales}
\begin{itemize}
	\item Desarrollar una aplicación para \emph{smartphone}.
	\item Implemenación y despliegue de la app en la tienda de de Google.
	\item Hacer que los usuarios pasen un buen rato.
	\item Mostrar quien es el creador de la aplicación, con el fin de poder usar esta como herramienta o portfolio.
	\item Desarrollar juegos.
\end{itemize}

\section{Objetivos técnicos}
\begin{itemize}
	\item Aprender una alternativa moderna a javascript mediante Dart.
	\item Comprender el funcionamiento de Flutter.
	\item Control de versiones con la herramienta GitHub, mediante comandos a través de GitBash.
	\item Generar documentación de todo el proceso en \LaTeX.
	\item Realizar una planificación mediante \emph{Scrum} eficiente, a través de la herramienta ZenHub, integrada en GitHub.
	\item Comenzar a usar las herramientas telemáticas para las reuniones con los tutores del trabajo final de grado. 
	\item Comunicación de la aplicación mediante WebServices.
	\item Conocer y aprender a usar las herramientas que proporciona el \emph{Cloud Services} de Google.
	\item Diseñar la arquitectura de la aplicación.
	
\end{itemize}

\section{Objetivos personales}
\begin{itemize}
	\item Adquirir el conocimiento necesario para desarrollar aplicaciones móviles y multiplataforma. Es decir, para tres entornos, Android, iOS y web.
	\item Aprobar el trabajo de fin de grado, ya que es un reto.
	\item Estudiar como generar documentación en \LaTeX.
	\item Aprender que sin esfuerzo no hay recompensa.
	\item Comprender el funcionamiento de la comunicación entre la una aplicación y los servicios en la nube.
	\item Reforzar los conocimientos adquiridos durante la carrera.
	\item Investigar diferentes herramientas para solventar los problemas que salgan.
	\item Adquirir las nociones necesarias para poder llevar proyectos.
	\item Conseguir manejarme en entornos de incertidumbre.
\end{itemize}